% Geometry package doesn't work with res margin option
%\usepackage[margin=1.0in]{geometry} 
%\topmargin=-0.5in
%\oddsidemargin -.5in
%\evensidemargin -.5in
%\textwidth=6.0in
%\itemsep=0in
%\parsep=0in

\addtolength{\textheight}{3cm}
%\addtolength{\voffset}{-1cm}

%\oddsidemargin=-1in
%\headsep=0pt
%\headheight=0pt
%\topmargin=-1in

%\usepackage[cm]{fullpage}
%\pagestyle{empty}

% the margin option causes section titles to appear to the left of body text 
\textwidth=5.2in % increase textwidth to get smaller right margin
% Font details here: http://www.tug.dk/FontCatalogue/lmodern/
%\usepackage{helvetica} % uses helvetica postscript font (download helvetica.sty)
%\usepackage{newcent}   % uses new century schoolbook postscript font 

\usepackage[T1]{fontenc}
\usepackage{lmodern}
%\usepackage[osf]{mathpazo}
%\renewcommand*\familydefault{\ttdefault}
%\renewcommand*\familydefault{\sfdefault} 

\usepackage{hyperref}
% No ugly borders
\hypersetup{
    colorlinks=false,
    pdfborder={0 0 0},
}

\usepackage{microtype}
\usepackage[english]{babel}


\usepackage{enumitem}
%\usepackage{layout}

% List and enumeration
\usepackage{enumitem}
\setitemize{itemsep=1pt,topsep=1pt,parsep=1pt,partopsep=1pt}


% Customizations
% From: https://www.sharelatex.com/blog/2011/03/27/how-to-write-a-latex-class-file-and-design-your-own-cv.html
\newcommand{\locdatesubsection}[3]{\textbf{#1} (#2) \hfill #3}
\newcommand{\nblocdatesubsection}[3]{#1 (#2) \hfill #3 }
\newcommand{\nbdatesubsection}[2]{#1 \hfill #2 }
\newcommand{\datesubsection}[2]{\textbf{#1} \hfill #2 }


% Multiline comments: http://tex.stackexchange.com/questions/87303/multi-line-block-comments-in-latex
%\long\def\/*#1*/{}

%\newcommand{\datedsubsection}[2]{%
%\subsection[#1]{#1 \hfill #2}%
%}

% Unbold description label: http://www.latex-community.org/forum/viewtopic.php?f=44&t=4204
%\renewcommand*\descriptionlabel[1]{\hspace\labelsep\itshape #1}


%\renewcommand\floatpagefraction{.9}
%\renewcommand\topfraction{.9}
%\renewcommand\bottomfraction{.9}
%\renewcommand\textfraction{.1}   
%\setcounter{totalnumber}{50}
%\setcounter{topnumber}{50}
%\setcounter{bottomnumber}{50}
