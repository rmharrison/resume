% LaTeX file for resume 
% This file uses the resume document class (res.cls)
% From http://www.rpi.edu/dept/arc/training/latex/resumes/

\documentclass[a4paper]{../res} 

% Geometry package doesn't work with res margin option
%\usepackage[margin=1.0in]{geometry} 
%\topmargin=-0.5in
%\oddsidemargin -.5in
%\evensidemargin -.5in
%\textwidth=6.0in
%\itemsep=0in
%\parsep=0in

%\usepackage[a4paper]{geometry}
\addtolength{\textheight}{3.5cm}
%\addtolength{\voffset}{-1cm}

%\oddsidemargin=-1in
%\headsep=0pt
%\headheight=0pt
%\topmargin=-1in

%\usepackage[cm]{fullpage}
%\pagestyle{empty}

% the margin option causes section titles to appear to the left of body text 
\textwidth=5.2in % increase textwidth to get smaller right margin
% Font details here: http://www.tug.dk/FontCatalogue/lmodern/
%\usepackage{helvetica} % uses helvetica postscript font (download helvetica.sty)
%\usepackage{newcent}   % uses new century schoolbook postscript font 

\usepackage[T1]{fontenc}
\usepackage{lmodern}
%\usepackage[osf]{mathpazo}
%\renewcommand*\familydefault{\ttdefault}
%\renewcommand*\familydefault{\sfdefault} 

\usepackage{hyperref}
% No ugly borders
\hypersetup{
    colorlinks=false,
    pdfborder={0 0 0},
}

\usepackage{microtype}
\usepackage[english]{babel}


\usepackage{enumitem}
%\usepackage{layout}

% List and enumeration
\usepackage{enumitem}
\setitemize{itemsep=1pt,topsep=1pt,parsep=1pt,partopsep=1pt}

% Customizations
% From: https://www.sharelatex.com/blog/2011/03/27/how-to-write-a-latex-class-file-and-design-your-own-cv.html
\newcommand{\locdatesubsection}[3]{\textbf{#1} (#2) \hfill #3}
%\newcommand{\nblocdatesubsection}[3]{#1 (#2) \hfill #3 }
\newcommand{\nblocdatesubsection}[3]{\vspace{-10pt}\section{\normalsize{#3}} {\normalsize#1} {\normalsize(#2)}}
\newcommand{\areasection}[1]{\section{\Large{#1}}\vspace{-6pt}\section{\noindent\rule{6.5in}{0.5pt}}\vspace{16pt}}
\newcommand{\nbdatesubsection}[2]{\vspace{-14pt}\section{\normalsize{#2}} #1 }
%\newcommand{\nbdatesubsection}[2]{#1 \hfill #1 }
\newcommand{\datesubsection}[2]{\textbf{#1} \hfill #2 }



% Multiline comments: http://tex.stackexchange.com/questions/87303/multi-line-block-comments-in-latex
%\long\def\/*#1*/{}

%\newcommand{\datedsubsection}[2]{%
%\subsection[#1]{#1 \hfill #2}%
%}

% Unbold description label: http://www.latex-community.org/forum/viewtopic.php?f=44&t=4204
%\renewcommand*\descriptionlabel[1]{\hspace\labelsep\itshape #1}


%\renewcommand\floatpagefraction{.9}
%\renewcommand\topfraction{.9}
%\renewcommand\bottomfraction{.9}
%\renewcommand\textfraction{.1}   
%\setcounter{totalnumber}{50}
%\setcounter{topnumber}{50}
%\setcounter{bottomnumber}{50}

\textwidth=6.5in

\begin{document} 
\begin{sloppypar}
 
%
\name{\textbf{\textsf{Ryan M Harrison}}\\ \\} % the \\[12pt] adds a blank line after name

%\address{2115 Cloville Avenue\\
%         Baltimore, MD 21214\\
%         \texttt{+}1 443 257 5953}
%\address{\texttt{ryan.m.harrison@gmail.com}\\
%         \texttt{linkedin.com/in/rmharri}\\
%         \texttt{github.com/rmharrison}}
%\address{Wolfson College\\ 
%         Oxford OX2 6UD\\
%         \texttt{+}44 07523 229446}

\address{2115 Cloville Avenue\\
         Baltimore, MD 21214\\
         +1 443 257 5953}
\address{ryan.m.harrison@gmail.com\\
         linkedin.com/in/rmharri\\
         github.com/rmharrison}
\address{Wolfson College\\ 
         Oxford OX2 6UD\\
         +44 (0) 7523 229446}

\address{2115 Cloville Avenue\\
         Baltimore, MD 21214\\
         +1 443 257 5953
         }
\address{\href{mailto:ryan.m.harrison@gmail.com}{\nolinkurl{ryan.m.harrison@gmail.com} } \\
         \href{http://linkedin.com/in/rmharri}{\nolinkurl{linkedin.com/in/rmharri}}\\
         \href{http://github.com/rmharrison}{\nolinkurl{github.com/rmharrison}}\\
        }
\address{Wolfson College\\ 
         Oxford OX2 6UD\\
         +44 (0) 7523 229446
         }


\begin{resume} 
Dowgate Hill \\
14-16 Dowgate Hill \\
London EC4R 2SU \\
United Kingdom \\
%\noindent\rule{5.2in}{0.75pt}
%\setlength{\parskip}{1.00ex}
%\setlength{\parindent}{0pt}
 
% Date
17 April 2014

% Inside address

%\vspace{1\baselineskip}

% Opening
\textbf{Re: Healthcare - Consultants - London (Cons-UK) via Oxford University Recruiting}

% Write the body of the letter
As a final year doctoral student in Computational Biophysics at the University of Oxford, I've honed my analytical and quantitative skills. The aim is not to abandon my training; rather, to gain a broader perspective and ultimately command considerably more leverage as a consultant, than I ever could as ``Just A Programmer.'' 

I am a highly international candidate; an American completing the doctorate in Oxford under the aegis of the National Institutes of Health (NIH-Oxford Scholar) and National Science Foundation (Graduate Research Fellow). In addition to the PhD abroad, I've also spent about a year studying and researching abroad, in Denmark and Japan. Prior to the PhD, I earned a degree in Biomedical Engineering from Johns Hopkins University.

Naturally, my training has synergy with the life sciences, and in addition to a plethora of academic research, I have interned with two life sciences companies. I am a competent bench scientist and have volunteered done emergency room volunteering.

The decision to pursue a career in consulting has not been taken lightly. Over the years, I've taken extensive coursework on business, economics and strategy (feel free to quiz me), worked on a team-based Oxford Student Consultancy project and reached out to PhD scientists that have successfully made the transition. They impressed upon me how potent the analytical and problem solving expertise of a doctorate can be when applied through the lens of business, a path I hope to follow. I'm committed to beginning my career in consulting.

I thought I'd share just one example of my personal impact. There are numerous technical examples, but I'm assuming that given my background, you'd be more interested in the non-technical. I took the opportunity to serve as a legislative aide to a Maryland State Legislature delegate representing 100k constituents. 90 days to run the affairs of a state numbering 6 million. 4 person legislative office where I'm the only full-time staffer.

I reworked the logistics backend to deliver mailed session status reports to 6k people, shepherded 3 bills from conception to passage and coordinated closely with stakeholders to the Insurance and Health Disparities subcommittees. The process exposed me to the complexities of the legislative process, revolving around the competing interest of doctors, hospitals, insurers, pharma and device manufacturers. None of this would have been possible without teamwork, effective communication and wholehearted dedication to the client, in this case the delegate and his constituents. More important than any individual outcome, I learned how to manage up, to best provide the delegate with actionable material, as well as laterally, to coordinate with legislative staff in other offices. I suspect these are the types of `soft skills' I'd use every day with Alcimed.


%Fairly or unfairly, DPhil applicants can be stereotyped as lacking in client management skills and knowledge of broader business ecosystems. To allay these concerns, I've worked to develop these skills outside of the research environment. 


%On the ecosystem side, I built foundational knowledge through coursework in econometrics, business strategy and corporate finance. This includes a business-focused study abroad in Denmark to better understand the Nordic model, an undergraduate capstone project ascertaining the appropriateness of different GDP measures and a science commercialization course at the Sa\"\i d Business School. To gain exposure to biotech in particular, I took the initiative and volunteered at exclusive industry conferences.

%As I've worked mostly with small research teams and start-ups, I've used this summer as an opportunity to learn more about the types of problems facing larger tech-focused firms, eventually landing in a R\&D role with NaturalMotion, a mid-sized games, animation and AI technology firm. The goal for this summer is to compare/contrast the small-scale environs I'm accustomed to, with the additional complications of running a larger firm. For example, the company is scaling very quickly and is increasingly struggling to maintain its culture; a problem class that I suspect faces many McKinsey clients after rapid organic growth or a large acquisition.

%\begin{description}[labelindent=1cm]
%    \item[Analytical] Handling scientific datasets from molecular simulation and image processing.
%\end{description}

%My value added is the ability to reduce ideas to practice, through both data analysis and coordination with key stakeholders. In the mid-term, I'm excited by the prospect of working with a diverse array of clients. In the long-term, I look forward to specializing where my technical background can be brought to bare on client's toughest challenges, for example with the Pharmaceuticals or Medical Technology practices.

Thank you for your time and consideration; I look forward to hearing from you.

%\vspace{1\baselineskip}
% Closing
Sincerely,
\\
Ryan M Harrison \\
Doctoral Candidate in Computational Biophysics, University of Oxford \\  \\
PS. Work authorization: US (indefinite), UK (Aug 2015, Tier 4 Doctorate Extension Scheme). \\ I would also be interested in the Princeton office.
%PS. Work authorization in the US and UK (Tier 4 Doctorate Extension Scheme).

\end{resume} 
\end{sloppypar}
\end{document} 



