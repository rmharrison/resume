% LaTeX file for resume 
% This file uses the resume document class (res.cls)
% From http://www.rpi.edu/dept/arc/training/latex/resumes/

\documentclass{../res} 

% Geometry package doesn't work with res margin option
%\usepackage[margin=1.0in]{geometry} 
%\topmargin=-0.5in
%\oddsidemargin -.5in
%\evensidemargin -.5in
%\textwidth=6.0in
%\itemsep=0in
%\parsep=0in

%\usepackage[a4paper]{geometry}
\addtolength{\textheight}{3.5cm}
%\addtolength{\voffset}{-1cm}

%\oddsidemargin=-1in
%\headsep=0pt
%\headheight=0pt
%\topmargin=-1in

%\usepackage[cm]{fullpage}
%\pagestyle{empty}

% the margin option causes section titles to appear to the left of body text 
\textwidth=5.2in % increase textwidth to get smaller right margin
% Font details here: http://www.tug.dk/FontCatalogue/lmodern/
%\usepackage{helvetica} % uses helvetica postscript font (download helvetica.sty)
%\usepackage{newcent}   % uses new century schoolbook postscript font 

\usepackage[T1]{fontenc}
\usepackage{lmodern}
%\usepackage[osf]{mathpazo}
%\renewcommand*\familydefault{\ttdefault}
%\renewcommand*\familydefault{\sfdefault} 

\usepackage{hyperref}
% No ugly borders
\hypersetup{
    colorlinks=false,
    pdfborder={0 0 0},
}

\usepackage{microtype}
\usepackage[english]{babel}


\usepackage{enumitem}
%\usepackage{layout}

% List and enumeration
\usepackage{enumitem}
\setitemize{itemsep=1pt,topsep=1pt,parsep=1pt,partopsep=1pt}

% Customizations
% From: https://www.sharelatex.com/blog/2011/03/27/how-to-write-a-latex-class-file-and-design-your-own-cv.html
\newcommand{\locdatesubsection}[3]{\textbf{#1} (#2) \hfill #3}
%\newcommand{\nblocdatesubsection}[3]{#1 (#2) \hfill #3 }
\newcommand{\nblocdatesubsection}[3]{\vspace{-10pt}\section{\normalsize{#3}} {\normalsize#1} {\normalsize(#2)}}
\newcommand{\areasection}[1]{\section{\Large{#1}}\vspace{-6pt}\section{\noindent\rule{6.5in}{0.5pt}}\vspace{16pt}}
\newcommand{\nbdatesubsection}[2]{\vspace{-14pt}\section{\normalsize{#2}} #1 }
%\newcommand{\nbdatesubsection}[2]{#1 \hfill #1 }
\newcommand{\datesubsection}[2]{\textbf{#1} \hfill #2 }



% Multiline comments: http://tex.stackexchange.com/questions/87303/multi-line-block-comments-in-latex
%\long\def\/*#1*/{}

%\newcommand{\datedsubsection}[2]{%
%\subsection[#1]{#1 \hfill #2}%
%}

% Unbold description label: http://www.latex-community.org/forum/viewtopic.php?f=44&t=4204
%\renewcommand*\descriptionlabel[1]{\hspace\labelsep\itshape #1}


%\renewcommand\floatpagefraction{.9}
%\renewcommand\topfraction{.9}
%\renewcommand\bottomfraction{.9}
%\renewcommand\textfraction{.1}   
%\setcounter{totalnumber}{50}
%\setcounter{topnumber}{50}
%\setcounter{bottomnumber}{50}

\textwidth=6.5in
\addtolength{\textheight}{-3cm}

\begin{document} 
\begin{sloppypar}
 
%
\name{\textbf{\textsf{Ryan M Harrison}}\\ \\} % the \\[12pt] adds a blank line after name

%\address{2115 Cloville Avenue\\
%         Baltimore, MD 21214\\
%         \texttt{+}1 443 257 5953}
%\address{\texttt{ryan.m.harrison@gmail.com}\\
%         \texttt{linkedin.com/in/rmharri}\\
%         \texttt{github.com/rmharrison}}
%\address{Wolfson College\\ 
%         Oxford OX2 6UD\\
%         \texttt{+}44 07523 229446}

\address{2115 Cloville Avenue\\
         Baltimore, MD 21214\\
         +1 443 257 5953}
\address{ryan.m.harrison@gmail.com\\
         linkedin.com/in/rmharri\\
         github.com/rmharrison}
\address{Wolfson College\\ 
         Oxford OX2 6UD\\
         +44 (0) 7523 229446}

\address{2115 Cloville Avenue\\
         Baltimore, MD 21214\\
         +1 443 257 5953
         }
\address{\href{mailto:ryan.m.harrison@gmail.com}{\nolinkurl{ryan.m.harrison@gmail.com} } \\
         \href{http://linkedin.com/in/rmharri}{\nolinkurl{linkedin.com/in/rmharri}}\\
         \href{http://github.com/rmharrison}{\nolinkurl{github.com/rmharrison}}\\
        }
\address{Wolfson College\\ 
         Oxford OX2 6UD\\
         +44 (0) 7523 229446
         }


\begin{resume} 
%\noindent\rule{5.2in}{0.75pt}
%\setlength{\parskip}{1.00ex}
%\setlength{\parindent}{0pt}
 
% Date
\vspace{1\baselineskip}
23 April 2014 \\

% Inside address
Mr. Edward Wack, Group Leader \\
Dr. Jeffrey Palmer, Associate Group Leader \\
Bioengineering Systems and Technologies Group \\
Lincoln Laboratory \\
Massachusetts Institute of Technology \\
244 Wood Street \\
Lexington, MA 02420-9108 \\
%\vspace{1\baselineskip}

% Opening
Mr. Wack \& Dr. Palmer:

\textbf{Re: Technical Staff -- Bio Informatics (4617)}

% Write the body of the letter
I'm an ideal fit for your group and I'm excited about working with you at Lincoln. Please take a moment to let me explain\ldots

My value to you and the Bioengineering group stems from three core competencies: 
\begin{itemize}[itemindent=1cm]
    \item Collaborative software development chops for prototyping (\texttt{Python}) and production (\texttt{C++})
    \item Extensive simulation method and data processing experience
    \item A detailed molecular understanding of DNA and proteins (theory and wetlab)
\end{itemize}
These core competencies are backed by practical industry experience and solid academics.

I will be relocating to Boston in August for a Fall 2014 start, after finishing up my doctorate in Computational Biophysics at the University of Oxford. Don't let the Oxford branding fool you, I'm a US citizen studying under the aegis of the NIH and NSF. Undergraduate degree in Biomedical Engineering from Johns Hopkins University.

To explore my interest in the group, I reached out to Dr. Peter Carr. A young growth area group with a collaborative model for technical staff to develop within; long-term projects with vision and direct applicability. Needless to say, I'm hooked. His enthusiasm and engaging descriptions certainly didn't hurt either.

I am a scientific software developer and competent bench scientist, best when working between wetware and software. For example, with a synthetic biology startup, where I developed plasmid assembly design software, then worked with the lab to integrate design, lab robotics and LIMS to build-out a complete idea-to-plasmid high-throughput assembly pipeline.

Another example is with a next-gen sequencing company, where I developed an analysis tool to pour over mounds of sequencing data to streamline the read-quality optimization process. As a bench scientist, I couldn't reduce the clearance time for the microfluidics, or improve the chip read-off time, but I could make the most of the data available.

As a DPhil student, I write tools to parse large-scale molecular simulation trajectories, piped in from high-performance CPU clusters running hundreds of simulations in parallel. The basic science of DNA behavior at high-curvature makes it interesting, but it's worthwhile because those insights will be directly applicable to experimentalist in the field of DNA nanotechnology. 

Wetware and software go hand-and-hand, and it takes a scientist with training in both to truly appreciate the beauty and utility of worlds that sometimes seem at odds.

Thank you for your time and attention. I look forward to hearing from you.

% Closing
Sincerely, \\ \\ \\
Ryan M Harrison \\ 
DPhil Candidate \\
Computational Biophysics \\
University of Oxford  \\ \\
Work authorization: US Citizen \\
Security clearance eligible, Level 1 NACI from NIH \\
\end{resume} 
\end{sloppypar}
\end{document} 



