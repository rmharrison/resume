% LaTeX file for resume 
% This file uses the resume document class (res.cls)
% From http://www.rpi.edu/dept/arc/training/latex/resumes/

\documentclass[a4paper]{../res} 
% Geometry package doesn't work with res margin option
%\usepackage[margin=1.0in]{geometry} 
%\topmargin=-0.5in
%\oddsidemargin -.5in
%\evensidemargin -.5in
%\textwidth=6.0in
%\itemsep=0in
%\parsep=0in

%\usepackage[a4paper]{geometry}
\addtolength{\textheight}{3.5cm}
%\addtolength{\voffset}{-1cm}

%\oddsidemargin=-1in
%\headsep=0pt
%\headheight=0pt
%\topmargin=-1in

%\usepackage[cm]{fullpage}
%\pagestyle{empty}

% the margin option causes section titles to appear to the left of body text 
\textwidth=5.2in % increase textwidth to get smaller right margin
% Font details here: http://www.tug.dk/FontCatalogue/lmodern/
%\usepackage{helvetica} % uses helvetica postscript font (download helvetica.sty)
%\usepackage{newcent}   % uses new century schoolbook postscript font 

\usepackage[T1]{fontenc}
\usepackage{lmodern}
%\usepackage[osf]{mathpazo}
%\renewcommand*\familydefault{\ttdefault}
%\renewcommand*\familydefault{\sfdefault} 

\usepackage{hyperref}
% No ugly borders
\hypersetup{
    colorlinks=false,
    pdfborder={0 0 0},
}

\usepackage{microtype}
\usepackage[english]{babel}


\usepackage{enumitem}
%\usepackage{layout}

% List and enumeration
\usepackage{enumitem}
\setitemize{itemsep=1pt,topsep=1pt,parsep=1pt,partopsep=1pt}

% Customizations
% From: https://www.sharelatex.com/blog/2011/03/27/how-to-write-a-latex-class-file-and-design-your-own-cv.html
\newcommand{\locdatesubsection}[3]{\textbf{#1} (#2) \hfill #3}
%\newcommand{\nblocdatesubsection}[3]{#1 (#2) \hfill #3 }
\newcommand{\nblocdatesubsection}[3]{\vspace{-10pt}\section{\normalsize{#3}} {\normalsize#1} {\normalsize(#2)}}
\newcommand{\areasection}[1]{\section{\Large{#1}}\vspace{-6pt}\section{\noindent\rule{6.5in}{0.5pt}}\vspace{16pt}}
\newcommand{\nbdatesubsection}[2]{\vspace{-14pt}\section{\normalsize{#2}} #1 }
%\newcommand{\nbdatesubsection}[2]{#1 \hfill #1 }
\newcommand{\datesubsection}[2]{\textbf{#1} \hfill #2 }



% Multiline comments: http://tex.stackexchange.com/questions/87303/multi-line-block-comments-in-latex
%\long\def\/*#1*/{}

%\newcommand{\datedsubsection}[2]{%
%\subsection[#1]{#1 \hfill #2}%
%}

% Unbold description label: http://www.latex-community.org/forum/viewtopic.php?f=44&t=4204
%\renewcommand*\descriptionlabel[1]{\hspace\labelsep\itshape #1}


%\renewcommand\floatpagefraction{.9}
%\renewcommand\topfraction{.9}
%\renewcommand\bottomfraction{.9}
%\renewcommand\textfraction{.1}   
%\setcounter{totalnumber}{50}
%\setcounter{topnumber}{50}
%\setcounter{bottomnumber}{50}

\textwidth=6.5in
%\addtolength{\textheight}{-3cm}

\begin{document} 
\begin{sloppypar}
 
%
\name{\textbf{\textsf{Ryan M Harrison}}\\ \\} % the \\[12pt] adds a blank line after name

%\address{2115 Cloville Avenue\\
%         Baltimore, MD 21214\\
%         \texttt{+}1 443 257 5953}
%\address{\texttt{ryan.m.harrison@gmail.com}\\
%         \texttt{linkedin.com/in/rmharri}\\
%         \texttt{github.com/rmharrison}}
%\address{Wolfson College\\ 
%         Oxford OX2 6UD\\
%         \texttt{+}44 07523 229446}

\address{2115 Cloville Avenue\\
         Baltimore, MD 21214\\
         +1 443 257 5953}
\address{ryan.m.harrison@gmail.com\\
         linkedin.com/in/rmharri\\
         github.com/rmharrison}
\address{Wolfson College\\ 
         Oxford OX2 6UD\\
         +44 (0) 7523 229446}

\address{2115 Cloville Avenue\\
         Baltimore, MD 21214\\
         +1 443 257 5953
         }
\address{\href{mailto:ryan.m.harrison@gmail.com}{\nolinkurl{ryan.m.harrison@gmail.com} } \\
         \href{http://linkedin.com/in/rmharri}{\nolinkurl{linkedin.com/in/rmharri}}\\
         \href{http://github.com/rmharrison}{\nolinkurl{github.com/rmharrison}}\\
        }
\address{Wolfson College\\ 
         Oxford OX2 6UD\\
         +44 (0) 7523 229446
         }


\begin{resume} 
%\vspace{1\baselineskip}

% Date
15 November 2013

% Inside address
Joanna Melewska \\
55 Baker Street \\
London W1U 8EW

% Opening
Dear Ms. Melewska,

\textbf{Re: Junior Consultant via Oxford recruiting, London Office}

I've had the pleasure of meeting with Roland Berger at the Oxford Consultancy Fair, where I had the opportunity to chat with you and Mr. Benjamin Murray. Even in our relatively brief interaction, I got the sense that Roland Berger people go above and beyond for the clients; for example, in you agreeing to look over my academic CV to adapt it to a more commercial-friendly style. I'm committed to beginning my career in consulting and think that Roland Berger would be a great place to start. 

As a final year DPhil in Computational Biophysics, I've honed my analytical skills in both the business and science arena. On the business side, I've worked at developing my business awareness through coursework, including an offering at the Sa\"\i d Business School. I then took the opportunity to apply what I'd learned through a team-based Oxford Student Consultancy project. This experience taught me valuable lessons about client interaction and reaffirmed my desire to begin a career in consulting. On the science-side, my thesis entails computational modelling of DNA. The details are unimportant; what matters is that, on a daily-basis, I draw conclusions from complex datasets and convey meaningful messages to a variety of stakeholders. With a colleague, this could be a white-board discussion about the appropriateness of a model assumption. With a senior professor, a high-level strategic overview and how my headline results interface with their work. Data is great, but it doesn't mean much without a mind to the audience and their, not my, needs. 

To be clear, the aim is not to abandon my training; rather, to gain a broader perspective and ultimately command considerably more leverage, as a consultant, than I ever could in the lab. To confirm this suspicion, I reached out to PhD scientists that have successfully navigated the transition from lab to consulting. They impressed upon me how potent the analytical and problem solving expertise of a doctorate can be when applied through the lens of business, a path I hope to follow. I look forward to interacting more with PhDs at Roland Berger, people like Dr. Duce Gotora, as I (hopefully) proceed through the application process.


%The life sciences practice would be a strong fit, given my life sciences background. While I intend to gain exposure to a diverse array of industries in the first few years; in the longer-term, I would value working with a practice that fully utilizes my doctorate and technical background, including:
%\begin{itemize}[itemindent=1cm]
%    \item Extensive knowledge of DNA and protein structure
%    \item Programming and analytical ability to test my hypothesise on real data
%    \item Experience with devices and instrumentation
%    \item Practical experience with doctors and patients through work and volunteering in hospital 
%\end{itemize}
%Why would I disregard these valuable experiences to fit into a consultancy that doesn't value life sciences sector expertise or the value-added of a doctorate. A compelling argument, and one of the reasons I'm so drawn to Roland Berger.

%One illustrative example of client/stakeholder management in a high-stakes environment is my work as a legislative aide to a Maryland State Legislature delegate, representing 100k constituents. I continued work on his anti-smoking legislation and coordinated closely with stakeholders to the Insurance and Health Disparities subcommittees. I was exposed to the complexities of the legislative process, revolving around the competing interest of doctors, hospitals, insurers, pharma and device manufacturers. More importantly, I learned how to manage up, to best provide the delegate with actionable material, as well as laterally, to coordinate with legislative staff in other offices.

Understandably, one may wonder what, other than raw analytical horsepower, a freshly minted DPhil can bring to the firm. One illustrative example of impact in a client-driven environment is my work as a legislative aide to a Maryland State Delegate, representing 100k constituents. Since I was the only full-time legislative staffer in an office of 4, there was much responsibility. I reworked the logistics backend to deliver mailed session status reports to 6k people. I shepherded 3 bills from conception to passage, all of which play their small part in keeping Maryland healthy. I coordinated with stakeholders to the Insurance and Health Disparities subcommittees, navigating the competing interest of doctors, hospitals, insurers, pharma and device manufacturers. None of this would have been possible without teamwork, effective communication and a wholehearted dedication to the client, in this case the delegate and his constituents. More important than any individual outcome, I learned how to manage up, to best provide the delegate with actionable material, as well as laterally, to coordinate with legislative staff in other offices. Hard-learned skills, I suspect, of value to a prospective consultant.

For Roland Berger, my value added is the ability to reduce ideas to practice, both through hardball analysis and the softer touch of stakeholder engagement. I'm excited by the prospect of working with a diverse array of clients before developing a specialism where my technical background can best be brought to bear on client's toughest challenges. Thank you for your time and consideration.

% Closing
Sincerely, \\
Ryan M Harrison \\
PS. Work authorization: US (indefinite), UK (Aug-2015 via Tier 4 Doctorate Extension Scheme). 


\end{resume} 
\end{sloppypar}
\end{document} 



