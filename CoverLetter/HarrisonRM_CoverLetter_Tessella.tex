% LaTeX file for resume 
% This file uses the resume document class (res.cls)
% From http://www.rpi.edu/dept/arc/training/latex/resumes/

\documentclass{../res} 

% Geometry package doesn't work with res margin option
%\usepackage[margin=1.0in]{geometry} 
%\topmargin=-0.5in
%\oddsidemargin -.5in
%\evensidemargin -.5in
%\textwidth=6.0in
%\itemsep=0in
%\parsep=0in

%\usepackage[a4paper]{geometry}
\addtolength{\textheight}{3.5cm}
%\addtolength{\voffset}{-1cm}

%\oddsidemargin=-1in
%\headsep=0pt
%\headheight=0pt
%\topmargin=-1in

%\usepackage[cm]{fullpage}
%\pagestyle{empty}

% the margin option causes section titles to appear to the left of body text 
\textwidth=5.2in % increase textwidth to get smaller right margin
% Font details here: http://www.tug.dk/FontCatalogue/lmodern/
%\usepackage{helvetica} % uses helvetica postscript font (download helvetica.sty)
%\usepackage{newcent}   % uses new century schoolbook postscript font 

\usepackage[T1]{fontenc}
\usepackage{lmodern}
%\usepackage[osf]{mathpazo}
%\renewcommand*\familydefault{\ttdefault}
%\renewcommand*\familydefault{\sfdefault} 

\usepackage{hyperref}
% No ugly borders
\hypersetup{
    colorlinks=false,
    pdfborder={0 0 0},
}

\usepackage{microtype}
\usepackage[english]{babel}


\usepackage{enumitem}
%\usepackage{layout}

% List and enumeration
\usepackage{enumitem}
\setitemize{itemsep=1pt,topsep=1pt,parsep=1pt,partopsep=1pt}

% Customizations
% From: https://www.sharelatex.com/blog/2011/03/27/how-to-write-a-latex-class-file-and-design-your-own-cv.html
\newcommand{\locdatesubsection}[3]{\textbf{#1} (#2) \hfill #3}
%\newcommand{\nblocdatesubsection}[3]{#1 (#2) \hfill #3 }
\newcommand{\nblocdatesubsection}[3]{\vspace{-10pt}\section{\normalsize{#3}} {\normalsize#1} {\normalsize(#2)}}
\newcommand{\areasection}[1]{\section{\Large{#1}}\vspace{-6pt}\section{\noindent\rule{6.5in}{0.5pt}}\vspace{16pt}}
\newcommand{\nbdatesubsection}[2]{\vspace{-14pt}\section{\normalsize{#2}} #1 }
%\newcommand{\nbdatesubsection}[2]{#1 \hfill #1 }
\newcommand{\datesubsection}[2]{\textbf{#1} \hfill #2 }



% Multiline comments: http://tex.stackexchange.com/questions/87303/multi-line-block-comments-in-latex
%\long\def\/*#1*/{}

%\newcommand{\datedsubsection}[2]{%
%\subsection[#1]{#1 \hfill #2}%
%}

% Unbold description label: http://www.latex-community.org/forum/viewtopic.php?f=44&t=4204
%\renewcommand*\descriptionlabel[1]{\hspace\labelsep\itshape #1}


%\renewcommand\floatpagefraction{.9}
%\renewcommand\topfraction{.9}
%\renewcommand\bottomfraction{.9}
%\renewcommand\textfraction{.1}   
%\setcounter{totalnumber}{50}
%\setcounter{topnumber}{50}
%\setcounter{bottomnumber}{50}

\textwidth=6.5in
\addtolength{\textheight}{-3cm}

\begin{document} 
\begin{sloppypar}
 
%
\name{\textbf{\textsf{Ryan M Harrison}}\\ \\} % the \\[12pt] adds a blank line after name

%\address{2115 Cloville Avenue\\
%         Baltimore, MD 21214\\
%         \texttt{+}1 443 257 5953}
%\address{\texttt{ryan.m.harrison@gmail.com}\\
%         \texttt{linkedin.com/in/rmharri}\\
%         \texttt{github.com/rmharrison}}
%\address{Wolfson College\\ 
%         Oxford OX2 6UD\\
%         \texttt{+}44 07523 229446}

\address{2115 Cloville Avenue\\
         Baltimore, MD 21214\\
         +1 443 257 5953}
\address{ryan.m.harrison@gmail.com\\
         linkedin.com/in/rmharri\\
         github.com/rmharrison}
\address{Wolfson College\\ 
         Oxford OX2 6UD\\
         +44 (0) 7523 229446}

\address{2115 Cloville Avenue\\
         Baltimore, MD 21214\\
         +1 443 257 5953
         }
\address{\href{mailto:ryan.m.harrison@gmail.com}{\nolinkurl{ryan.m.harrison@gmail.com} } \\
         \href{http://linkedin.com/in/rmharri}{\nolinkurl{linkedin.com/in/rmharri}}\\
         \href{http://github.com/rmharrison}{\nolinkurl{github.com/rmharrison}}\\
        }
\address{Wolfson College\\ 
         Oxford OX2 6UD\\
         +44 (0) 7523 229446
         }


\begin{resume} 
%\noindent\rule{5.2in}{0.75pt}
%\setlength{\parskip}{1.00ex}
%\setlength{\parindent}{0pt}
 
% Date
23 March 2014

% Inside address
Grant Stephen, CEO (USA) \\
Tessella Inc \\
233 Needham Street, Suite 300 \\
Newton, MA 02464
%\vspace{1\baselineskip}

% Opening
Dear Mr. Stephen: %

\textbf{Re: Graduate Scientific Software Developer}

% Write the body of the letter
As a final year doctoral candidate in Computational Biophysics at the University of Oxford, I've honed my analytical and quantitative skills through years of training. With Tessella, I want to apply this hard earned skill set to solving industry's toughest challenges. 

Given my background, the life sciences practice in Boston seems particularly well suited. To confirm this suspicion, I reached out to Damayanti Gupta, a pharmaceutical and biotech specialist. She kindly entertained my questions about the firm, impressed upon me my suitability for the role and encouraged me to apply.

My value to Tessella stems from three core competencies: 
\begin{itemize}[itemindent=1cm]
    \item Collaborative software development experience in C++
    \item Extensive simulation method and data processing experience
    \item A demonstrated interest in business
\end{itemize}

These core competencies are backed by both a solid academic foundation and practical industry experience. I have served as a developer for two large-scale pieces of academic software, the Rosetta Protein Structure Prediction Suite and the oxDNA coarse-grained DNA model. Realizing that scientific code is very different than commercial code, I took last summer as an opportunity to intern with NaturalMotion, a game animation middleware and mobile games company based in Oxford. They wanted someone with simulation experience to work with a newly developed numerical solver for the procedural animation of complex game characters. Having previously worked with small companies, I wanted to learn about Agile development in a more corporate environment. 

I took the lessons of NaturalMotion back to oxDNA, where I pushed for a refactor of our analysis toolchain, implemented auto-documentation and built an integration test suite. These tools have improved usability and dramatically reduced the occurrence of bugs. This is but one example of my personal impact in a technical setting; specifically, the interplay between my academic and industry work.

It should be emphasized that this technical ability is paired with an ability to work in a non-technical setting; for example, as a legislative aide to the Maryland General Assembly. The result is a switch-hitter, comfortable with both the deeply technical and the deeply non-technical.

Thank you for your time and attention. I look forward to hearing from you.

% Closing
Sincerely, \\
Ryan M Harrison \\ 
DPhil Candidate, Computational Biophysics, University of Oxford  \\
PS. Work authorization: US (indefinite), UK (Aug 2015, Tier 4 Doctorate Extension Scheme).
\end{resume} 
\end{sloppypar}
\end{document} 



