% LaTeX file for resume 
% This file uses the resume document class (res.cls)
% From http://www.rpi.edu/dept/arc/training/latex/resumes/

\documentclass{../res} 

% Geometry package doesn't work with res margin option
%\usepackage[margin=1.0in]{geometry} 
%\topmargin=-0.5in
%\oddsidemargin -.5in
%\evensidemargin -.5in
%\textwidth=6.0in
%\itemsep=0in
%\parsep=0in

%\usepackage[a4paper]{geometry}
\addtolength{\textheight}{3.5cm}
%\addtolength{\voffset}{-1cm}

%\oddsidemargin=-1in
%\headsep=0pt
%\headheight=0pt
%\topmargin=-1in

%\usepackage[cm]{fullpage}
%\pagestyle{empty}

% the margin option causes section titles to appear to the left of body text 
\textwidth=5.2in % increase textwidth to get smaller right margin
% Font details here: http://www.tug.dk/FontCatalogue/lmodern/
%\usepackage{helvetica} % uses helvetica postscript font (download helvetica.sty)
%\usepackage{newcent}   % uses new century schoolbook postscript font 

\usepackage[T1]{fontenc}
\usepackage{lmodern}
%\usepackage[osf]{mathpazo}
%\renewcommand*\familydefault{\ttdefault}
%\renewcommand*\familydefault{\sfdefault} 

\usepackage{hyperref}
% No ugly borders
\hypersetup{
    colorlinks=false,
    pdfborder={0 0 0},
}

\usepackage{microtype}
\usepackage[english]{babel}


\usepackage{enumitem}
%\usepackage{layout}

% List and enumeration
\usepackage{enumitem}
\setitemize{itemsep=1pt,topsep=1pt,parsep=1pt,partopsep=1pt}

% Customizations
% From: https://www.sharelatex.com/blog/2011/03/27/how-to-write-a-latex-class-file-and-design-your-own-cv.html
\newcommand{\locdatesubsection}[3]{\textbf{#1} (#2) \hfill #3}
%\newcommand{\nblocdatesubsection}[3]{#1 (#2) \hfill #3 }
\newcommand{\nblocdatesubsection}[3]{\vspace{-10pt}\section{\normalsize{#3}} {\normalsize#1} {\normalsize(#2)}}
\newcommand{\areasection}[1]{\section{\Large{#1}}\vspace{-6pt}\section{\noindent\rule{6.5in}{0.5pt}}\vspace{16pt}}
\newcommand{\nbdatesubsection}[2]{\vspace{-14pt}\section{\normalsize{#2}} #1 }
%\newcommand{\nbdatesubsection}[2]{#1 \hfill #1 }
\newcommand{\datesubsection}[2]{\textbf{#1} \hfill #2 }



% Multiline comments: http://tex.stackexchange.com/questions/87303/multi-line-block-comments-in-latex
%\long\def\/*#1*/{}

%\newcommand{\datedsubsection}[2]{%
%\subsection[#1]{#1 \hfill #2}%
%}

% Unbold description label: http://www.latex-community.org/forum/viewtopic.php?f=44&t=4204
%\renewcommand*\descriptionlabel[1]{\hspace\labelsep\itshape #1}


%\renewcommand\floatpagefraction{.9}
%\renewcommand\topfraction{.9}
%\renewcommand\bottomfraction{.9}
%\renewcommand\textfraction{.1}   
%\setcounter{totalnumber}{50}
%\setcounter{topnumber}{50}
%\setcounter{bottomnumber}{50}

\textwidth=6.5in
\textheight 11in %RMH
\addtolength{\textheight}{-3cm}

\begin{document} 
\begin{sloppypar}
 
%
\name{\textbf{\textsf{Ryan M Harrison}}\\ \\} % the \\[12pt] adds a blank line after name

%\address{2115 Cloville Avenue\\
%         Baltimore, MD 21214\\
%         \texttt{+}1 443 257 5953}
%\address{\texttt{ryan.m.harrison@gmail.com}\\
%         \texttt{linkedin.com/in/rmharri}\\
%         \texttt{github.com/rmharrison}}
%\address{Wolfson College\\ 
%         Oxford OX2 6UD\\
%         \texttt{+}44 07523 229446}

\address{2115 Cloville Avenue\\
         Baltimore, MD 21214\\
         +1 443 257 5953}
\address{ryan.m.harrison@gmail.com\\
         linkedin.com/in/rmharri\\
         github.com/rmharrison}
\address{Wolfson College\\ 
         Oxford OX2 6UD\\
         +44 (0) 7523 229446}

\address{2115 Cloville Avenue\\
         Baltimore, MD 21214\\
         +1 443 257 5953
         }
\address{\href{mailto:ryan.m.harrison@gmail.com}{\nolinkurl{ryan.m.harrison@gmail.com} } \\
         \href{http://linkedin.com/in/rmharri}{\nolinkurl{linkedin.com/in/rmharri}}\\
         \href{http://github.com/rmharrison}{\nolinkurl{github.com/rmharrison}}\\
        }
\address{Wolfson College\\ 
         Oxford OX2 6UD\\
         +44 (0) 7523 229446
         }


\begin{resume} 
%\noindent\rule{5.2in}{0.75pt}
%\setlength{\parskip}{1.00ex}
%\setlength{\parindent}{0pt}
 
% Date
5 May 2014

% Inside address
Victor Velculescu, M.D. Ph.D., CSO \\
Personal Genome Diagnostics, Inc. \\
2809 Boston St, Suite 503 \\
Baltimore, MD 21224 \\
%\vspace{1\baselineskip}

% Opening
Dr. Velculescu:

\textbf{Re: Genomics Analyst / Associate Scientist}

% Write the body of the letter
When I learned about Personal Genome through Johns Hopkins University career services (ID: 60357), I was estatic. We're a great fit, and here's why\ldots

My value to Personal Genome stems from three core competencies: 
\begin{itemize}[itemindent=1cm]
    \item Collaborative software development experience supplemented with extensive simulation method and data processing experience
    \item Detailed molecular understanding of DNA, including a stint developing a pre-release next-generation sequencer
    \item A demonstrated interest in the life sciences ecosystem
\end{itemize}

These core competencies are backed by both a solid academic foundation and practical industry experience. I have served as a developer for two large-scale pieces of academic software, the Rosetta Protein Structure Prediction Suite and the oxDNA coarse-grained DNA model. I've also worked with 3 companies, one of which was Ion Torrent Systems, builders of the Personal Genome Machine. As such, I'm familiar with next-gen sequencing approaches, Ion in particular at a hardware and read-quality determination level. 

I'll be relocating back to the US in August 2014, after submitting my thesis. Currently finishing up a doctorate Computational Biophysics at the University of Oxford, under the aegis and generous fellowship support of the NSF and NIH. In fact, as an NIH graduate partnership student completing a portion of my dissertation work at the NIH, I was a contemporary of one of your current scientist, Dr. Sonya Parpart.

On a personal note, I'm a Baltimore native; proud product of Baltimore City Schools and Johns Hopkins University Whiting School of Engineering. Baltimore isn't exactly a life sciences hub, and I wasn't expecting to identify a firm with such a perfect fit outside of hubs like Boston. I'm glad I've been shown lacking, in my faith in the city's start-up landscape.

I would be very excited to further discuss my fit with the firm. Thank you for your time and attention. I look forward to hearing from you.

% Closing
Sincerely, \\
Ryan M Harrison \\ 
DPhil Candidate \\
Computational Biophysics \\
University of Oxford  \\
PS. Work authorization: US (indefinite), UK (Aug 2015, Tier 4 Doctorate Extension Scheme).
\end{resume} 
\end{sloppypar}
\end{document} 

%Note upon JHU submission (ID: 60357)
%Baltimore native and DNA extraordinary ecstatic to learn about your firm. Finishing up a doctorate in Computational Biophysics (molecular simulation of DNA) at the University of Oxford. Relocating back to the states in August 2014. Probably more suitable for an Associate Scientist position, but I just couldn't pass up the opportunity to potentially work with a Baltimore-based biotech start-up.

%Note that at Oxford, doctorates are called DPhils, and are entirely research based. No transcript.

%No surprise\ldotsINTP. Surprisingly well balanced between Math and Verbal on Standardized tests, e.g. 700/V and 670/M on SAT (back when it was out of 1600 total).
